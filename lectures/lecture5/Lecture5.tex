\input{../utils/preamble}
\createdgmtitle{5}
%--------------------------------------------------------------------------------
\begin{document}
%--------------------------------------------------------------------------------
\begin{frame}[noframenumbering,plain]
%\thispagestyle{empty}
\titlepage
\end{frame}
%=======
\begin{frame}{Recap of previous lecture}
	\begin{block}{Jacobian matrix}
		Let $f: \mathbb{R}^m \rightarrow \mathbb{R}^m$ be a differentiable function.
		\[
		\bz = f(\bx), \quad 
		\bJ =  \frac{\partial \bz}{\partial \bx} =
		\begin{pmatrix}
			\frac{\partial z_1}{\partial x_1} & \dots & \frac{\partial z_1}{\partial x_m} \\
			\dots & \dots & \dots \\ 
			\frac{\partial z_m}{\partial x_1} & \dots & \frac{\partial z_m}{\partial x_m}
		\end{pmatrix} \in \bbR^{m \times m}
		\]
		\vspace{-0.3cm}
	\end{block}
	\begin{block}{Change of variable theorem (CoV)}
		Let $\bx$ be a random variable with density function $p(\bx)$ and $f: \mathbb{R}^m \rightarrow \mathbb{R}^m$ is a differentiable, invertible function (diffeomorphism). If $\bz = f(\bx)$, $\bx = f^{-1}(\bz) = g(\bz)$, then
		\begin{align*}
			p(\bx) &= p(\bz) |\det(\bJ_f)| = p(\bz) \left|\det \left(  \frac{\partial \bz}{\partial \bx} \right) \right| = p(f(\bx)) \left|\det \left(  \frac{\partial f(\bx)}{\partial \bx} \right) \right| \\
			p(\bz) &= p(\bx) |\det(\bJ_g)|= p(\bx) \left|\det \left(  \frac{\partial \bx}{\partial \bz} \right) \right| = p(g(\bz)) \left|\det \left(  \frac{\partial g(\bz)}{\partial \bz} \right) \right|.
		\end{align*}
		\vspace{-0.5cm}
	\end{block}
\end{frame}
%=======
\begin{frame}{Recap of previous lecture}
	\begin{block}{Definition}
		Normalizing flow is a \textit{differentiable, invertible} mapping from data $\bx$ to the noise $\bz$. 
	\end{block}
	\vspace{-0.1cm}
	\begin{figure}
		\includegraphics[width=0.85\linewidth]{figs/flows_how2}
	\end{figure}
	\vspace{-0.5cm}
	\begin{block}{Log likelihood}
		\vspace{-0.5cm}
		\[
			\log p(\bx | \btheta) = \log p(f_K \circ \dots \circ f_1(\bx)) + \sum_{k=1}^K\log |\det (\bJ_{f_k})|
		\]
	\end{block}
	\myfootnotewithlink{https://arxiv.org/abs/1605.08803}{Dinh L., Sohl-Dickstein J., Bengio S. Density estimation using Real NVP, 2016} 
\end{frame}
%=======
\begin{frame}{Recap of previous lecture}
	\begin{block}{Forward KL for flow model}
	  	\vspace{-0.1cm}
		\[
			\log p(\bx|\btheta) = \log p(f(\bx, \btheta)) + \log  |\det (\bJ_f)|
		\]
		\vspace{-0.3cm}
	\end{block}
	\begin{block}{Reverse KL for flow model}
  		\vspace{-0.5cm}
		\[
			KL(p || \pi)  = \bbE_{p(\bz)} \left[  \log p(\bz) -  \log |\det (\bJ_g)| - \log \pi(g(\bz, \btheta)) \right]
		\]
		\vspace{-0.5cm}
	\end{block}
	\begin{block}{Flow KL duality}
	  	\vspace{-0.3cm}
		\[
			\argmin_{\btheta} KL(\pi(\bx) || p(\bx | \btheta)) = \argmin_{\btheta} KL(p(\bz | \btheta) || p(\bz))
		\]
		\vspace{-0.3cm}
		\begin{itemize}
			\item $p(\bz)$ is a base distribution; $\pi(\bx)$ is a data distribution;
			\item $\bz \sim p(\bz)$, $\bx = g(\bz, \btheta)$, $\bx \sim p(\bx| \btheta)$;
			\item $\bx \sim \pi(\bx)$, $\bz = f(\bx, \btheta)$, $\bz \sim p(\bz | \btheta)$.
		\end{itemize}
	\end{block}
	\myfootnotewithlink{https://arxiv.org/abs/1912.02762}{Papamakarios G. et al. Normalizing flows for probabilistic modeling and inference, 2019} 
\end{frame}
%=======
\begin{frame}{Recap of previous lecture}
	\vspace{-0.5cm}
	\begin{block}{Flow log-likelihood}
		\vspace{-0.3cm}
		\[
			\log p(\bx|\btheta) = \log p(f(\bx, \btheta)) + \log  |\det (\bJ_f)|
		\]
		\vspace{-0.5cm}
	\end{block}
	The main challenge is a determinant of the Jacobian.
	\begin{block}{Linear flows}	
		\vspace{-0.2cm}
		\[
			\bz = f(\bx, \btheta) = \bW \bx, \quad \bW \in \bbR^{m \times m}, \quad \btheta = \bW, \quad \bJ_f = \bW^T
		\]
	\end{block}
	\vspace{-0.3cm}
	\begin{itemize}
		\item LU-decomposition
		\[
			\bW = \mathbf{P} \bL \bU.
		\]
		\item QR-decomposition
		\[
			\bW = \bQ \mathbf{R}.
		\]
	\end{itemize}
	Decomposition should be done only once in the beggining. Next, we fit decomposed matrices ($\bP/\bL/\bU$ or $\bQ/\bR$).
	\myfootnote{\href{https://arxiv.org/abs/1807.03039}{Kingma D. P., Dhariwal P. Glow: Generative Flow with Invertible 1x1 Convolutions, 2018}  \\
	\href{https://arxiv.org/abs/1901.11137}{Hoogeboom E., et al. Emerging convolutions for generative normalizing flows, 2019}
	}
\end{frame}
%=======
\begin{frame}{Recap of previous lecture}
	Consider an autoregressive model
	\vspace{-0.3cm}
	{\small
		\[
		p(\bx | \btheta) = \prod_{j=1}^m p(x_j | \bx_{1:j - 1}, \btheta), \quad
		p(x_j | \bx_{1:j - 1}, \btheta) = \mathcal{N} \left(\mu_j(\bx_{1:j-1}), \sigma^2_j (\bx_{1:j-1})\right).
		\]
	}
	\vspace{-0.5cm}
	\begin{block}{Gaussian autoregressive NF}
		\vspace{-0.5cm}
		\begin{align*}
			\bx &= g(\bz, \btheta) \quad \Rightarrow \quad {\color{violet} x_j} = \sigma_j ({\color{violet} \bx_{1:j-1}}) \cdot {\color{teal} z_j} + \mu_j({\color{violet} \bx_{1:j-1}}). \\
			\bz &= f(\bx, \btheta) \quad \Rightarrow \quad {\color{teal} z_j} = \left({\color{violet}x_j} - \mu_j({\color{violet}\bx_{1:j-1}}) \right) \cdot \frac{1}{ \sigma_j ({\color{violet}\bx_{1:j-1}})}.
		\end{align*}
		\vspace{-0.5cm}
	\end{block}
	\begin{itemize}
		\item We have an \textbf{invertible} and \textbf{differentiable} transformation from $p(\bz)$ to $p(\bx | \btheta)$.
		\item Jacobian of such transformation is triangular!
	\end{itemize}
	Generation function $g(\bz, \btheta)$ is \textbf{sequential}. \\ Inference function $f(\bx, \btheta)$ is \textbf{not sequential}.
	
	\myfootnotewithlink{https://arxiv.org/abs/1705.07057}{Papamakarios G., Pavlakou T., Murray I. Masked Autoregressive Flow for Density Estimation, 2017} 
\end{frame}
%=======
\begin{frame}{Gaussian autoregressive NF}
	\vspace{-0.5cm}
	\begin{align*}
		\bx &= g(\bz, \btheta) \quad \Rightarrow \quad {\color{violet} x_j} = \sigma_j ({\color{violet} \bx_{1:j-1}}) \cdot {\color{teal} z_j} + \mu_j({\color{violet} \bx_{1:j-1}}). \\
		\bz &= f(\bx, \btheta) \quad \Rightarrow \quad {\color{teal} z_j} = \left({\color{violet}x_j} - \mu_j({\color{violet}\bx_{1:j-1}}) \right) \cdot \frac{1}{ \sigma_j ({\color{violet}\bx_{1:j-1}})}.
	\end{align*}
	
	\begin{itemize}
		\item Sampling is sequential, density estimation is parallel.
		\item Forward KL is a natural loss.
	\end{itemize}
	\vspace{-0.3cm}
	\begin{minipage}[t]{0.65\columnwidth}
		\begin{block}{Forward transform: $g(\bz, \btheta)$}
			\[
			x_j = \sigma_j (\bx_{1:j-1}) \cdot z_j + \mu_j(\bx_{1:j-1})
			\]
		\end{block}
	\end{minipage}%
	\begin{minipage}[t]{0.35\columnwidth}
		\begin{figure}[h]
			\centering
			\includegraphics[width=.9\linewidth]{figs/af_iaf_explained_1.png}
		\end{figure}
	\end{minipage} \\
	
	\begin{minipage}[t]{0.65\columnwidth}
		\begin{block}{Inverse transform: $f(\bx, \btheta)$}
			\vspace{-0.5cm}
			\[
				z_j = (x_j - \mu_j(\bx_{1:j-1})) \cdot \frac{1}{\sigma_j (\bx_{1:j-1}) }
			\]
			\vspace{-0.4cm}
		\end{block}
	\end{minipage}% 
	\begin{minipage}[t]{0.35\columnwidth}
		\begin{figure}[h]
			\centering
			\includegraphics[width=.9\linewidth]{figs/af_iaf_explained_2.png}
		\end{figure}
	\end{minipage}
	
	\myfootnotewithlink{https://blog.evjang.com/2018/01/nf2.html}{image credit: https://blog.evjang.com/2018/01/nf2.html}
\end{frame}
\begin{frame}{Outline}
	\tableofcontents
\end{frame}
%=======
\section{RealNVP: coupling layer}
%=======
\begin{frame}{RealNVP}
	\vspace{-0.5cm}
	Let split $\bx$ and $\bz$ in two parts: 
	\[
		\bx = [\bx_1, \bx_2] = [\bx_{1:d}, \bx_{d+1:m}]; \quad \bz = [\bz_1, \bz_2] = [\bz_{1:d}, \bz_{d+1:m}].
	\]
	\vspace{-0.7cm}
	\begin{block}{Coupling layer}
		\vspace{-0.7cm}
		\[
			\begin{cases} \bx_1 = \bz_1; \\ \bx_2 = \bz_2 \odot \bsigma(\bz_1, \btheta) + \bmu(\bz_1, \btheta).\end{cases}  
			\begin{cases} \bz_1 = \bx_1; \\ \bz_2 = \left(\bx_2 - \bmu({\color{olive}\bx_1}, \btheta) \right) \odot \frac{1}{\bsigma({\color{olive}\bx_1}, \btheta)}.\end{cases}
		\]
	\end{block}
	\vspace{-0.5cm}
	\begin{block}{Image partitioning}
		
		\begin{minipage}[t]{0.5\columnwidth}
			\begin{figure}
				\centering
				\includegraphics[width=\linewidth]{figs/realnvp_masking.png}
			\end{figure}
		\end{minipage}% 
		\begin{minipage}[t]{0.5\columnwidth}
			\begin{itemize}
				\item Checkerboard ordering uses masking.
				\item Channelwise ordering uses splitting.
			\end{itemize}
		\end{minipage}
	\end{block}
	\vspace{-0.5cm}
	\myfootnotewithlink{https://arxiv.org/abs/1605.08803}{Dinh L., Sohl-Dickstein J., Bengio S. Density estimation using Real NVP, 2016} 
\end{frame}
%=======
\begin{frame}{RealNVP}
	\begin{block}{Coupling layer}
		\vspace{-0.7cm}
		\[
		 \begin{cases} {\color{violet}\bx_1} = {\color{teal}\bz_1}; \\ {\color{violet}\bx_2} = {\color{teal}\bz_2} \odot \bsigma({\color{teal}\bz_1}, \btheta) + \bmu({\color{teal}\bz_1}, \btheta).\end{cases}  
		\begin{cases} {\color{teal}\bz_1} ={\color{violet} \bx_1}; \\ {\color{teal}\bz_2} = \left({\color{violet}\bx_2} - \bmu({\color{violet}\bx_1}, \btheta) \right) \odot \frac{1}{\bsigma({\color{violet}\bx_1}, \btheta)}.\end{cases}
		\]
		Estimating the density takes 1 pass, sampling takes 1 pass!
	\end{block}
	\begin{block}{Jacobian}
		\vspace{-0.5cm}
		\[
		\det \left( \frac{\partial \bz}{\partial \bx} \right) = \det 
		\begin{pmatrix}
			\bI_d & 0_{d \times m - d} \\
			\frac{\partial \bz_2}{\partial \bx_1} & \frac{\partial \bz_2}{\partial \bx_2}
		\end{pmatrix} = \prod_{j=1}^{m - d} \frac{1}{\sigma_j(\bx_1, \btheta)}.
		\]
		\vspace{-0.5cm}
	\end{block}
	\begin{block}{Gaussian AR NF}
		\vspace{-0.6cm}
		\begin{align*}
			\bx &= g(\bz, \btheta) \quad \Rightarrow \quad {\color{violet}x_j} = \sigma_j ({\color{violet}\bx_{1:j-1}}) \cdot {\color{teal} z_j} + \mu_j({\color{violet}\bx_{1:j-1}}). \\
			\bz &= f(\bx, \btheta) \quad \Rightarrow \quad {\color{teal} z_j} = \left({\color{violet} x_j} - \mu_j({\color{violet}\bx_{1:j-1}}) \right) \cdot \frac{1}{\sigma_j ({\color{violet} \bx_{1:j-1}}) }.
		\end{align*}
		\vspace{-0.5cm}
	\end{block}
	How to get RealNVP coupling layer from gaussian AR NF?
	
	\myfootnotewithlink{https://arxiv.org/abs/1605.08803}{Dinh L., Sohl-Dickstein J., Bengio S. Density estimation using Real NVP, 2016} 
\end{frame}
%=======
\begin{frame}{Glow samples}
	Glow model: coupling layer + linear flows (1x1 convs)
	\begin{figure}
		\centering
		\includegraphics[width=0.9\linewidth]{figs/glow_faces.png}
	\end{figure}
	\myfootnotewithlink{https://arxiv.org/abs/1807.03039}{Kingma D. P., Dhariwal P. Glow: Generative Flow with Invertible 1x1 Convolutions, 2018}
\end{frame}
%=======
\begin{frame}{Venn diagram for Normalizing flows}
	
	\begin{figure}
		\centering
		\includegraphics[width=\linewidth]{figs/venn_diagram}
	\end{figure}
	\begin{itemize}
		\item $\cI$ -- invertible functions.
		\item $\cF$ -- continuously differentiable functions whose Jacobian is lower triangular.
		\item $\cM$ -- invertible functions from $\cF$.
	\end{itemize}
	\myfootnotewithlink{https://arxiv.org/abs/1907.07945}{Song Y., Meng C., Ermon S. Mintnet: Building invertible neural networks with masked convolutions, 2019}
\end{frame}
%=======
\section{Normalizing flows as VAE model}
%=======
\begin{frame}{VAE vs Normalizing flows}
	\begin{table}[]
		\begin{tabular}{l|c|c}
			& \textbf{VAE} & \textbf{NF} \\ \hline
			\textbf{Objective} & ELBO $\cL$ & Forward KL/MLE \\ \hline
			\textbf{Encoder} & \shortstack{stochastic \\ $\bz \sim q (\bz | \bx, \bphi)$} &  \shortstack{\\ deterministic \\ $\bz = f(\bx, \btheta)$ \\ $q(\bz | \bx, \btheta) = \delta(\bz - f(\bx, \btheta))$}  \\ \hline
			\textbf{Decoder} & \shortstack{stochastic \\ $\bx \sim p (\bx | \bz, \btheta)$} & \shortstack{\\ deterministic \\ $\bx = g(\bz, \btheta)$ \\ $ p(\bx | \bz, \btheta) = \delta(\bx - g(\bz, \btheta))$} \\ \hline
			\textbf{Parameters}  & $\bphi, \btheta$ & $\btheta \equiv \bphi$\\ 
		\end{tabular}
	\end{table}
	\begin{block}{Theorem}
		MLE for normalizing flow is equivalent to maximization of ELBO for VAE model with deterministic encoder and decoder:
		\vspace{-0.3cm}
		\[
			p(\bx | \bz, \btheta) = \delta (\bx - f^{-1}(\bz, \btheta)) = \delta (\bx - g(\bz, \btheta));
		\]
		\[
			q(\bz | \bx, \btheta) = p(\bz | \bx, \btheta) = \delta (\bz - f(\bx, \btheta)).
		\]
	\end{block}
	\myfootnotewithlink{https://arxiv.org/abs/2007.02731}{Nielsen D., et al. SurVAE Flows: Surjections to Bridge the Gap between VAEs and Flows, 2020}
\end{frame}
%=======
\begin{frame}{Normalizing flow as VAE}
	\begin{block}{Proof}
		\begin{enumerate}
			\item Dirac delta function property 
			\[
				\bbE_{\delta(\bx - \by)} f(\bx) = \int \delta(\bx - \by) f(\bx) d \bx = f(\by).
			\]
			\item CoV theorem and Bayes theorem:
			\[
				p(\bx | \btheta) = p(\bz) |\det (\bJ_f)|;
			\]
			\[
				p(\bz | \bx, \btheta) = \frac{p(\bx | \bz, \btheta) p(\bz)}{p(\bx | \btheta)}; \quad \Rightarrow \quad p(\bx | \bz, \btheta) = p(\bz | \bx, \btheta) |\det (\bJ_f)|.
			\]
			\item Log-likelihood decomposition
			\[
				\log p(\bx | \btheta) = \cL(\btheta) + {\color{olive}KL(q(\bz | \bx, \btheta) || p(\bz | \bx, \btheta))} = \cL(\btheta).
			\]
		\end{enumerate}
	\end{block}
	\myfootnotewithlink{https://arxiv.org/abs/2007.02731}{Nielsen D., et al. SurVAE Flows: Surjections to Bridge the Gap between VAEs and Flows, 2020}
\end{frame}
%=======
\begin{frame}{Normalizing flow as VAE}
	\begin{block}{Proof}
		ELBO objective:
		\vspace{-0.5cm}
		\begin{multline*}
			\cL  = \bbE_{q(\bz | \bx, \btheta)} \left[\log p(\bx | \bz, \btheta) - \log \frac{q(\bz | \bx, \btheta)}{p(\bz)} \right]  \\
			= \bbE_{q(\bz | \bx, \btheta)} \left[{\color{violet}\log \frac{p(\bx | \bz, \btheta)}{q(\bz | \bx, \btheta)}} + {\color{teal}\log p(\bz)} \right].
		\end{multline*}
		\vspace{-0.6cm}
		\begin{enumerate}
			\item  Dirac delta function property:
			\vspace{-0.3cm}
			\[
				{\color{teal}\bbE_{q(\bz | \bx, \btheta)} \log p(\bz)} = \int \delta (\bz - f(\bx, \btheta)) \log p(\bz) d \bz = \log p(f(\bx, \btheta)).
			\]
			\vspace{-0.6cm}
			\item CoV theorem and Bayes theorem:
			\vspace{-0.2cm}
			{ \small
			\[ 
				{\color{violet}\bbE_{q(\bz | \bx, \btheta)} \log \frac{p(\bx | \bz, \btheta)}{q(\bz | \bx, \btheta)}} = \bbE_{q(\bz | \bx, \btheta)} \log \frac{p(\bz | \bx, \btheta) |\det (\bJ_f)|}{q(\bz | \bx, \btheta)} = \log |\det \bJ_f|.
			\]
			}
			\vspace{-0.6cm}
			\item Log-likelihood decomposition
			\vspace{-0.3cm}
			\[
				\log p(\bx | \btheta) = \cL(\btheta) = \log p(f(\bx, \btheta)) +  \log |\det \bJ_f|.
			\]
		\end{enumerate}
	\end{block}
	\myfootnotewithlink{https://arxiv.org/abs/2007.02731}{Nielsen D., et al. SurVAE Flows: Surjections to Bridge the Gap between VAEs and Flows, 2020}
\end{frame}
%=======
\section{Discrete data vs continuous model}
%=======
\begin{frame}{Discrete data vs continuous model}
	Let our data $\by$ comes from discrete distribution $\Pi(\by)$ and we have continuous model $p(\bx | \btheta) = \text{NN}(\bx, \btheta)$.
	\begin{itemize}
		\item Images {\color{gray}(and not only images)} are discrete data, pixels lie in the integer domain (\{0, 255\}). 
		\item By fitting a continuous density model $p(\bx | \btheta)$ to discrete data $\Pi(\by)$, one can produce a degenerate solution with all probability mass on discrete values. 
	\end{itemize}
	\begin{block}{Discrete model}
		\begin{itemize}
			\item Use \textbf{discrete} model (e.x. $P(\by | \btheta) = \text{Cat}(\bpi(\btheta))$). 
			\item Minimize any suitable divergence measure $D(\Pi, P)$.
			\item NF works only with continuous data $\bx$ (there are discrete NF, see papers below).
			\item If pixel value is not presented in the train data, it won't be predicted.		
		\end{itemize}
	\end{block}
	\myfootnote{\href{https://arxiv.org/abs/1905.07376}{Hoogeboom E. et al. Integer discrete flows and lossless compression, 2019} \\
		\href{https://arxiv.org/abs/1905.10347}{Tran D. et al. Discrete flows: Invertible generative models of discrete data, 2019}}
\end{frame}
%=======
\begin{frame}{Discrete data vs continuous model}
	\begin{block}{Continuous model}
		\begin{itemize}
			\item Use \textbf{continuous} model (e.x. $p(\bx | \btheta) = \cN(\bmu_{\btheta}(\bx), \bsigma_{\btheta}^2(\bx))$), but
			\begin{itemize}
				\item \textbf{discretize} model (make the model outputs discrete): transform $p(\bx | \btheta)$ to $P(\by | \btheta)$;
				\item \textbf{dequantize} data (make the data continuous): transform $\Pi(\by)$ to $\pi(\bx)$.
			\end{itemize}
			\item Continuous distribution knows numerical relationships.
		\end{itemize}
	\end{block}
	\begin{block}{CIFAR-10 pixel values distribution}
		\begin{figure}
			\includegraphics[width=0.6\linewidth,height=0.2\linewidth]{figs/CIFAR_pixel_distr}
		\end{figure}
	\end{block}
	\myfootnotewithlink{https://arxiv.org/abs/1701.05517}{Salimans T. et al. PixelCNN++: Improving the PixelCNN with Discretized Logistic Mixture Likelihood and Other Modifications, 2017}
\end{frame}
%=======
\subsection{Discretization of continuous distribution}
%=======
\begin{frame}{Discretization of continuous distribution}
	\vspace{-0.3cm}
	\begin{block}{Model discretization through CDF}
		\vspace{-0.6cm}
		\[
			F(\bx | \btheta) = \int_{-\infty}^{\bx} p(\bx' | \btheta) d\bx'; \quad 
			P(\by | \btheta) = F(\by + 0.5 | \btheta) - F(\by - 0.5 | \btheta)
		\]
	\end{block}
	\vspace{-0.6cm}
	\begin{block}{Mixture of logistic distributions}
		\vspace{-0.7cm}
		\[
			p(x | \mu, s) = \frac{\exp^{-(x - \mu) / s}}{s (1 + \exp^{-(x - \mu) / s})^2}; \quad p(x | \bpi, \bmu, \bs) = \sum_{k=1}^K \pi_k p(x | \mu_k, s_k).
		\]
		\vspace{-0.7cm}
	\end{block}
	\begin{block}{PixelCNN++}
		\vspace{-0.7cm}
		\[
			p(\bx | \btheta) = \prod_{j=1}^m p(x_j | \bx_{1:j-1}, \btheta); \quad p(x_j | \bx_{1:j-1}, \btheta) = \sum_{k=1}^K \pi_k p(x | \mu_k, s_k).
		\]
		\vspace{-0.5cm} \\
		Here, $\pi_k = \pi_{k, \btheta}(\bx_{1:j-1})$, $\mu_k = \mu_{k, \btheta}(\bx_{1:j-1})$, $s_k = s_{k, \btheta}(\bx_{1:j-1})$.
	\end{block}

	For the pixel edge cases of 0, replace $y - 0.5$ by $-\infty$, and for 255 replace $y + 0.5$ by $+\infty$.
	\myfootnotewithlink{https://arxiv.org/abs/1701.05517}{Salimans T. et al. PixelCNN++: Improving the PixelCNN with Discretized Logistic Mixture Likelihood and Other Modifications, 2017}\end{frame}
%=======
\subsection{Dequantization of discrete data}
%=======
\begin{frame}{Uniform dequantization}
	Let dequantize discrete distribution $\Pi(\by)$ to continuous distribution $\pi(\bx)$ in the following way: $\bx = \by + \bu$, where  $\bu \sim U[0, 1]$.
	\begin{minipage}{0.7\linewidth}	
		\begin{block}{Theorem}
			Fitting continuous model $p(\bx | \btheta)$ on uniformly dequantized data is equivalent to maximization of a lower bound on log-likelihood for a discrete model:
			\vspace{-0.2cm}
			\[
			P(\by | \btheta) = \int_{U[0, 1]} p(\by + \bu | \btheta) d \bu
			\]
			\vspace{-0.5cm} 
		\end{block}
	\end{minipage}%
	\begin{minipage}{0.3\linewidth}
		\begin{figure}
			\centering
			\includegraphics[width=\linewidth,height=0.8\linewidth]{figs/uniform_dequantization.png}
		\end{figure}
	\end{minipage}

	\begin{block}{Proof}
		\vspace{-0.8cm}
		{\small
		\begin{multline*}
			\bbE_{\pi} \log p(\bx | \btheta) = \int \pi(\bx) \log p(\bx | \btheta) d \bx = \sum \Pi(\by) \int_{U[0,1]} \log p(\by + \bu | \btheta) d \bu \leq \\
			\leq \sum \Pi(\by) \log \int_{U[0,1]}  p(\by + \bu | \btheta) d \bu = \\ = \sum \Pi(\by) \log P(\by | \btheta) = \bbE_{\Pi} \log P(\by | \btheta).
		\end{multline*}
		}
	\end{block}
	\myfootnotewithlink{https://arxiv.org/abs/1511.01844}{Theis L., Oord A., Bethge M. A note on the evaluation of generative models. 2015}
\end{frame}
%=======
\begin{frame}{Variational dequantization}
	\begin{minipage}[t]{0.5\columnwidth}
		\begin{figure}
			\centering
			\includegraphics[width=0.8\linewidth]{figs/uniform_dequantization.png}
		\end{figure}
	\end{minipage}%
	\begin{minipage}[t]{0.5\columnwidth}
		\begin{figure}
			\centering
			\includegraphics[width=0.8\linewidth]{figs/variational_dequantization.png}
		\end{figure}
	\end{minipage}
	\begin{itemize}
		\item $p(\bx | \btheta)$ assign uniform density to unit hypercubes $\by + U[0, 1]$ (left fig).
		\item Smooth dequantization is more natural (right fig).
		\item Neural network density models are smooth function approximators.
	\end{itemize}
	Introduce variational dequantization noise distribution $q(\bu | \by)$, which tells what kind of noise we have to add to our discrete data.  \\
	Treat it as an approximate posterior as in VAE model. 
\end{frame}
%=======
\begin{frame}{Variational dequantization}
	\vspace{-0.2cm}
	\begin{block}{Variational lower bound}
		\vspace{-0.7cm}
		\begin{multline*}
			\log P(\by | \btheta) = \left[ \log \int q(\bu | \by) \frac{p(\by + \bu | \btheta)}{q(\bu | \by)} d \bu \right] \geq \\ 
			\geq  \int q(\bu | \by) \log \frac{p(\by + \bu | \btheta)}{q(\bu | \by)} d \bu = \mathcal{L}(q, \btheta).
		\end{multline*}
		\vspace{-0.6cm}
	\end{block}
	Uniform dequantization is a special case of variational dequantization ($q(\bu | \by) = U[0, 1]$).
	\begin{block}{Flow++: flow-based variational dequantization}
		Let $\bu = g(\bepsilon, \by, \blambda)$ is a flow model with base distribution $\bepsilon \sim p(\bepsilon)$:
		\vspace{-0.3cm}
		\[
		q(\bu | \by) = p(f(\bu, \by, \blambda)) \cdot \left| \det \frac{\partial f(\bu, \by, \blambda)}{\partial \bu}\right|.
		\]
		\vspace{-0.3cm}
		\[
		\log P(\by | \btheta) \geq \cL(\blambda, \btheta) = \int p(\bepsilon) \log \left( \frac{p(\by + g(\bepsilon, \by, \blambda) | \btheta)}{p(\bepsilon) \cdot \left| \det \bJ_g\right|^{-1}} \right) d\bepsilon.
		\]
		\vspace{-0.3cm}
	\end{block}
	\myfootnotewithlink{https://arxiv.org/abs/1902.00275}{Ho J. et al. Flow++: Improving Flow-Based Generative Models with Variational Dequantization and Architecture Design, 2019}
\end{frame}
%=======
\begin{frame}{Summary}
	\begin{itemize}
		\item Gaussian autoregressive flow is an autoregressive model with triangular Jacobian. It has fast inference function and slow generation function. Forward KL is a natural loss function.
		\vfill
		\item The RealNVP coupling layer is an effective type of flow (special case of AR flows) that has fast inference and generation modes.
		\vfill
		\item NF models could be treated as VAE model with deterministic encoder and decoder.
		\vfill
		\item Lots of data are discrete. We able to discretize the model or to dequantize our data to use continuous model.
		\vfill
		\item Uniform dequantization is the simplest form of dequantization. Variational dequantization is a more natural type that uses variational inference.
	\end{itemize}
\end{frame}
%=======
\end{document} 