\input{../utils/preamble}
\createdgmtitle{10}
%--------------------------------------------------------------------------------
\begin{document}
%--------------------------------------------------------------------------------
\begin{frame}[noframenumbering,plain]
%\thispagestyle{empty}
\titlepage
\end{frame}
%=======
\begin{frame}{Recap of previous lecture}
	\vspace{-0.3cm}
	\begin{block}{f-divergence minimization}
		\vspace{-0.3cm}
		\[
			D_f(\pi || p) = \bbE_{p(\bx)}  f\left( \frac{\pi(\bx)}{p(\bx)} \right) \rightarrow \min_p.
		\]
		Here $f: \bbR_+ \rightarrow \bbR$ is a convex, lower semicontinuous function satisfying $f(1) = 0$.
	\end{block}
	\begin{block}{Variational divergence estimation}
		\vspace{-0.3cm}
		\[
			D_f(\pi || p) \geq \sup_{T \in \cT} \left[\bbE_{\pi}T(\bx) -  \bbE_p f^*(T(\bx)) \right],
		\]
		\vspace{-0.7cm}
	\end{block}
	\begin{block}{Fenchel conjugate}
		\vspace{-0.7cm}
		\[
		f^*(t) = \sup_{u \in \text{dom}_f} \left( ut - f(u) \right), \quad f(u) = \sup_{t \in \text{dom}_{f^*}} \left( ut - f^*(t) \right)
		\]
		\vspace{-0.5cm}
	\end{block}
	\textbf{Note:} To evaluate lower bound we only need samples from $\pi(\bx)$ and $p(\bx)$. Hence, we could fit implicit generative model.
	\myfootnotewithlink{https://arxiv.org/abs/1606.00709}{Nowozin S., Cseke B., Tomioka R. f-GAN: Training Generative Neural Samplers using Variational Divergence Minimization, 2016}
\end{frame}
%=======
\begin{frame}{Recap of previous lecture}
	How to evaluate likelihood-free models?
	
	$p(y | \bx)$ -- pretrained image classification model (e.g. ImageNet classifier).
	\begin{block}{What do we want from samples?}
		\begin{itemize}
			\item \textbf{Sharpness}
			\begin{figure}
				\centering
				\includegraphics[width=0.9\linewidth]{figs/sharpness}
			\end{figure}
			$p(y | \bx)$ has low entropy (each image $\bx$ should have distinctly recognizable object).
			\item \textbf{Diversity}
			\begin{figure}
				\centering
				\includegraphics[width=0.9\linewidth]{figs/diversity}
			\end{figure}
			$p(y) = \int p(y | \bx) p(\bx) d \bx$ has high entropy (there should be as many classes generated as possible).
		\end{itemize}
	\end{block}
	\myfootnotewithlink{https://deepgenerativemodels.github.io}{image credit: https://deepgenerativemodels.github.io}
\end{frame}
%=======
\begin{frame}{Recap of previous lecture}
	Let take some pretrained image classification model to get the conditional label distribution $p(y | \bx)$ (e.g. ImageNet classifier).
	\begin{block}{Evaluation of likelihood-free models}
		\begin{itemize}
			\item Sharpness $\Rightarrow$ low $H(y | \bx) = - \sum_{y} \int_{\bx} p(y, \bx) \log p(y | \bx) d\bx$.
			\item Diversity $\Rightarrow$ high $H(y)  = - \sum_{y} p(y) \log p(y)$.
		\end{itemize}
	\end{block}
	\vspace{-0.3cm}
	\begin{block}{Inception Score}
		\vspace{-0.2cm}
		\[
			IS = \exp(H(y) - H(y | \bx)) = \exp \left( \bbE_{\bx} KL(p(y | \bx) || p(y)) \right)
		\]
		\vspace{-0.5cm}
	\end{block}
	\begin{block}{Frechet Inception Distance}
		\vspace{-0.3cm}
		\[
			D^2 (\pi, p) = \| \mathbf{m}_{\pi} - \mathbf{m}_{p}\|_2^2 + \text{Tr} \left( \bSigma_{\pi} + \bSigma_p - 2 \sqrt{\bSigma_{\pi} \bSigma_p} \right).
		\]
		\vspace{-0.5cm}
	\end{block}
	FID is related to moment matching.
	
	\myfootnote{\href{https://arxiv.org/abs/1606.03498}{Salimans T. et al. Improved Techniques for Training GANs, 2016} \\
	\href{https://arxiv.org/abs/1706.08500}{Heusel M. et al. GANs Trained by a Two Time-Scale Update Rule Converge to a Local Nash Equilibrium, 2017} }
\end{frame}
%=======
\begin{frame}{Recap of previous lecture}
		\begin{itemize}
			\item $\cS_{\pi} = \{\bx_i\}_{i=1}^{n} \sim \pi(\bx)$ -- real samples;
			\item $\cS_{p} = \{\bx_i\}_{i=1}^{n} \sim p(\bx | \btheta)$ -- generated samples.
		\end{itemize}
		Embed samples using pretrained classifier network (as previously):
		\[
			\cG_{\pi} = \{\mathbf{g}_i\}_{i=1}^n, \quad \cG_{p} = \{\mathbf{g}_i\}_{i=1}^n.
		\]
		Define binary function:
		\[
			f(\mathbf{g}, \cG) = 
			\begin{cases}
				1, \text{if exists } \mathbf{g}' \in \cG: \| \mathbf{g}  - \mathbf{g}'\|_2 \leq \| \mathbf{g}' - \text{NN}_k(\mathbf{g}', \cG)\|_2; \\
				0, \text{otherwise.}
			\end{cases}
		\]
		\[
			\text{Precision} (\cG_{\pi}, \cG_{p}) = \frac{1}{n} \sum_{\mathbf{g} \in \cG_{p}} f(\mathbf{g}, \cG_{\pi}); \quad \text{Recall} (\cG_{\pi}, \cG_{p}) = \frac{1}{n} \sum_{\mathbf{g} \in \cG_{\pi}} f(\mathbf{g}, \cG_{p}).
		\]
		\vspace{-0.4cm}
		\begin{figure}
			\includegraphics[width=0.7\linewidth]{figs/pr_k_nearest}
		\end{figure}
		\myfootnotewithlink{https://arxiv.org/abs/1904.06991}{Kynkäänniemi T. et al. Improved precision and recall metric for assessing generative models, 2019}
\end{frame}
%=======
\begin{frame}{Outline}
	\tableofcontents
\end{frame}
%=======
\section{Neural ODE}
%=======
\begin{frame}{Neural ODE}
	Consider Ordinary Differential Equation (ODE)
	\begin{align*}
	    \frac{d \bz(t)}{dt} &= f_{\btheta}(\bz(t), t); \quad \text{with initial condition }\bz(t_0) = \bz_0. \\
	    \bz(t_1) &= \int^{t_1}_{t_0} f_{\btheta}(\bz(t), t) d t  + \bz_0 = \text{ODESolve}(\bz(t_0), f_{\btheta}, t_0,t_1).
	\end{align*}
	\vspace{-0.4cm}
	\begin{block}{Euler update step}
		\vspace{-0.6cm}
		\[
		    \frac{\bz(t + \Delta t) - \bz(t)}{\Delta t} = f_{\btheta}(\bz(t), t) \,\, \Rightarrow \,\, \bz(t + \Delta t) = \bz(t) + \Delta t \cdot f_{\btheta}(\bz(t), t)
		\]
		\vspace{-0.7cm}
	\end{block}
	\begin{block}{Residual block}
		\begin{minipage}[t]{0.7\columnwidth}
			\vspace{-0.4cm}
			\[
				\bz_{t + 1} = \bz_t + f_{\btheta}(\bz_t)
			\]
			\vspace{-0.6cm}
			\begin{itemize}
				 \item It is equavalent to Euler update step for solving ODE with $\Delta t = 1$!
				 \item Euler update step is unstable and trivial. There are more sophisticated methods.
			\end{itemize}
		\end{minipage}%
		\begin{minipage}[t]{0.3\columnwidth}
			\vspace{-0.2cm}
			\begin{figure}
			    \centering
			    \includegraphics[width=\linewidth]{figs/resnet_1.png}
			\end{figure}
		\end{minipage}
		\vspace{-0.4cm}
	\end{block}

	\myfootnotewithlink{https://arxiv.org/abs/1806.07366}{Chen R. T. Q. et al. Neural Ordinary Differential Equations, 2018}   
\end{frame}
%=======
\begin{frame}{Neural ODE}
	\begin{block}{Residual block}
	\vspace{-0.4cm}
	\[
	    \bz_{t+1} = \bz_t + f_{\btheta}(\bz_t).
	\]
	\vspace{-0.4cm}
	\end{block}
	In the limit of adding more layers and taking smaller steps, we parameterize the continuous dynamics of hidden units using an ODE specified by a neural network: 
	\[
	    \frac{d \bz(t)}{dt} = f_{\btheta}(\bz(t), t); \quad \bz(t_0) = \bx; \quad \bz(t_1) = \by.
	\]
	\begin{minipage}[t]{0.4\columnwidth}
		\begin{figure}
			\centering
			\includegraphics[width=0.8\linewidth]{figs/euler}
		\end{figure}
	\end{minipage}%
	\begin{minipage}[t]{0.6\columnwidth}
		\vspace{-0.4cm}
		\begin{figure}
			\centering
			\includegraphics[width=0.9\linewidth]{figs/resnet_vs_neural_ode.png}
		\end{figure}
	\end{minipage}

	\myfootnotewithlink{https://arxiv.org/abs/1806.07366}{Chen R. T. Q. et al. Neural Ordinary Differential Equations, 2018}   
\end{frame}
%=======
\begin{frame}{Neural ODE}
	\begin{block}{Forward pass (loss function)}
		\vspace{-0.8cm}
		\begin{align*}
			L(\by) = L(\bz(t_1)) &= L\left( \bz(t_0) + \int_{t_0}^{t_1} f_{\btheta}(\bz(t), t) dt \right) \\ &= L\bigl(\text{ODESolve}(\bz(t_0), f_{\btheta}, t_0,t_1) \bigr)
		\end{align*}
	\vspace{-0.5cm}
	\end{block}
	\textbf{Note:} ODESolve could be any method (Euler step, Runge-Kutta methods).
	\begin{block}{Backward pass (gradients computation)}
		For fitting parameters we need gradients:
		\[
			\ba_{\bz}(t) = \frac{\partial L(\by)}{\partial \bz(t)}; \quad \ba_{\btheta}(t) = \frac{\partial L(\by)}{\partial \btheta(t)}.
		\]
		In theory of optimal control these functions called \textbf{adjoint} functions. 
		They show how the gradient of the loss depends on the hidden state~$\bz(t)$ and parameters $\btheta$.
	\end{block}

	\myfootnotewithlink{https://arxiv.org/abs/1806.07366}{Chen R. T. Q. et al. Neural Ordinary Differential Equations, 2018}     
\end{frame}
%=======
\section{Adjoint method}
%=======
\begin{frame}{Adjoint method}
	\vspace{-0.3cm}
	\begin{block}{Adjoint functions}
		\vspace{-0.3cm}
		\[
		\ba_{\bz}(t) = \frac{\partial L(\by)}{\partial \bz(t)}; \quad \ba_{\btheta}(t) = \frac{\partial L(\by)}{\partial \btheta(t)}.
		\]
		\vspace{-0.6cm}
	\end{block}
	\begin{block}{Theorem (Pontryagin)}
		\vspace{-0.6cm}
		\[
		\frac{d \ba_{\bz}(t)}{dt} = - \ba_{\bz}(t)^T \cdot \frac{\partial f_{\btheta}(\bz(t), t)}{\partial \bz}; \quad \frac{d \ba_{\btheta}(t)}{dt} = - \ba_{\bz}(t)^T \cdot \frac{\partial f_{\btheta}(\bz(t),  t)}{\partial \btheta}.
		\]
		Do we know any initilal condition?
	\end{block}
	\begin{block}{Solution for adjoint function}
		\vspace{-0.8cm}
		\begin{align*}
			\frac{\partial L}{\partial \btheta(t_0)} &= \ba_{\btheta}(t_0) =  - \int_{t_1}^{t_0} \ba_{\bz}(t)^T \frac{\partial f_{\btheta}(\bz(t), t)}{\partial \btheta(t)} dt + 0\\
			\frac{\partial L}{\partial \bz(t_0)} &= \ba_{\bz}(t_0) =  - \int_{t_1}^{t_0} \ba_{\bz}(t)^T \frac{\partial f_{\btheta}(\bz(t), t)}{\partial \bz(t)} dt + \frac{\partial L}{\partial \bz(t_1)}\\
		\end{align*}
		\vspace{-1.2cm}
	\end{block}
	\textbf{Note:} These equations are solved back in time.
	\myfootnotewithlink{https://arxiv.org/abs/1806.07366}{Chen R. T. Q. et al. Neural Ordinary Differential Equations, 2018}   
\end{frame}
%=======
\begin{frame}{Adjoint method}
	\vspace{-0.2cm}
	\begin{block}{Forward pass}
		\vspace{-0.5cm}
		\[
		\bz(t_1) = \int^{t_1}_{t_0} f_{\btheta}(\bz(t), t) d t  + \bz_0 \quad \Rightarrow \quad \text{ODE Solver}
		\]
		\vspace{-0.6cm}
	\end{block}
	\begin{block}{Backward pass}
		\vspace{-0.8cm}
		\begin{equation*}
			\left.
			{\footnotesize 
				\begin{aligned}
					\frac{\partial L}{\partial \btheta(t_0)} &= \ba_{\btheta}(t_0) =  - \int_{t_1}^{t_0} \ba_{\bz}(t)^T \frac{\partial f_{\btheta}(\bz(t), t)}{\partial \btheta(t)} dt + 0 \\
					\frac{\partial L}{\partial \bz(t_0)} &= \ba_{\bz}(t_0) =  - \int_{t_1}^{t_0} \ba_{\bz}(t)^T \frac{\partial f_{\btheta}(\bz(t), t)}{\partial \bz(t)} dt + \frac{\partial L}{\partial \bz(t_1)} \\
					\bz(t_0) &= - \int^{t_0}_{t_1} f_{\btheta}(\bz(t), t) d t  + \bz_1.
				\end{aligned}
			}
			\right\rbrace
			\Rightarrow
			\text{ODE Solver}
		\end{equation*}
		\vspace{-0.4cm} 
	\end{block}
	\textbf{Note:} These scary formulas are the standard backprop in the discrete case.
	\begin{figure}
		\centering
		\includegraphics[width=\linewidth]{figs/neural_ode}
	\end{figure}
	\myfootnotewithlink{https://arxiv.org/abs/1806.07366}{Chen R. T. Q. et al. Neural Ordinary Differential Equations, 2018}   
\end{frame}
%=======
\section{Continuous-in-time Normalizing Flows}
%=======
\begin{frame}{Continuous-in-time Normalizing Flows}
	\vspace{-0.3cm}
	\begin{block}{Discrete-in-time NF}
		\vspace{-0.8cm}
		  \[
		  \bz_{t+1} = f_{\btheta}(\bz_t); \quad \log p(\bz_{t+1}) = \log p(\bz_{t}) - \log \left| \det \frac{\partial f_{\btheta}(\bz_t)}{\partial \bz_{t}} \right| .
		  \]
		\vspace{-0.7cm}
	\end{block}
	\begin{block}{Continuous-in-time dynamics}
		\vspace{-0.2cm}
		\[
			\frac{d\bz(t)}{dt} = f_{\btheta}(\bz(t), t).
		\]
	\end{block}
	\vspace{-0.6cm}
	\begin{minipage}[t]{0.4\columnwidth}
		\begin{figure}
			\centering
			\includegraphics[width=0.75\linewidth]{figs/cnf_flow.png}
		\end{figure}
	\end{minipage}%
	\begin{minipage}[t]{0.6\columnwidth}
		\begin{figure}
			\centering
			\includegraphics[width=0.8\linewidth]{figs/ffjord.png}
		\end{figure}
	\end{minipage}
	\myfootnotewithlink{https://arxiv.org/abs/1810.01367}{Grathwohl W. et al. FFJORD: Free-form Continuous Dynamics for Scalable Reversible Generative Models, 2018}  
\end{frame}
%=======
\begin{frame}{Continuous-in-time Normalizing Flows}
	\begin{block}{Theorem (Picard)}
		If $f$ is uniformly Lipschitz continuous in $\bz$ and continuous in $t$, then the ODE has a \textbf{unique} solution.
	\end{block}
	\textbf{Note:} Unlike discrete-in-time NF, $f$ does not need to be bijective (uniqueness guarantees bijectivity).
	\begin{itemize}
		\item Discrete-in-time NF need invertible $f$. Here we have sequence of $\log p(\bz_t)$.
		\item Continuous-in-time NF require only smoothness of $f$. Here we need to get $\log(p(\bz(t), t))$
	\end{itemize}
	\begin{block}{Forward and inverse transforms}
		\vspace{-0.7cm}
		\begin{align*}
			\bx &= \bz(t_1) = \bz(t_0) + \int_{t_0}^{t_1} f_{\btheta}(\bz(t), t) dt \\
			\bz &= \bz(t_0) = \bz(t_1) + \int_{t_1}^{t_0} f_{\btheta}(\bz(t), t) dt
		\end{align*}
		\vspace{-0.7cm}
	\end{block}
	\myfootnotewithlink{https://arxiv.org/abs/1806.07366}{Chen R. T. Q. et al. Neural Ordinary Differential Equations, 2018}   
\end{frame}
%=======
\begin{frame}{Continuous-in-time Normalizing Flows}
	\vspace{-0.3cm}
	\begin{block}{Theorem (Kolmogorov-Fokker-Planck: special case)}
		If $f$ is uniformly Lipschitz continuous in $\bz$ and continuous in $t$, then
		\[
			\frac{d \log p(\bz(t), t)}{d t} = - \text{tr} \left( \frac{\partial f_{\btheta}(\bz(t), t)}{\partial \bz(t)} \right).
		\]
	\end{block}
	\vspace{-0.6cm}
	\[
		\log p(\bx | \btheta) = \log p(\bz) - \int_{t_0}^{t_1} \text{tr}  \left( \frac{\partial f_{\btheta}(\bz(t), t)}{\partial \bz(t)} \right) dt.
	\]
	Here $p(\bx | \btheta) = p(\bz(t_1), t_1)$, $p(\bz) = p(\bz(t_0), t_0)$.
	\textbf{Adjoint} method is used for getting the derivatives.
	\begin{block}{Forward transform + log-density}
		\vspace{-0.3cm}
		\[
			\begin{bmatrix}
				\bx \\
				\log p(\bx | \btheta)
			\end{bmatrix}
			= 
			\begin{bmatrix}
				\bz \\
				\log p(\bz)
			\end{bmatrix} + 
			\int_{t_0}^{t_1} 
			\begin{bmatrix}
				f_{\btheta}(\bz(t), t) \\
				- \text{tr} \left( \frac{\partial f_{\btheta}(\bz(t), t)}{\partial \bz(t)} \right) 
			\end{bmatrix} dt.
		\]
		\vspace{-0.4cm}
	\end{block}
	It costs $O(m^2)$ to get the trace of the Jacobian (evaluation of determinant of the Jacobian costs $O(m^3)$!).
	\myfootnotewithlink{https://arxiv.org/abs/1806.07366}{Chen R. T. Q. et al. Neural Ordinary Differential Equations, 2018}   
\end{frame}
%=======
\begin{frame}{Continuous-in-time Normalizing Flows}
	\vspace{0.2cm}
	\begin{itemize}
		\item $\text{tr} \left( \frac{\partial f_{\btheta}(\bz(t))}{\partial \bz(t)} \right)$ costs $O(m^2)$ ($m$
		evaluations of $f$), since we have to compute a derivative for each diagonal element. 
		\item Jacobian vector products ${\color{violet}\bv^T \frac{\partial f}{\partial \bz}}$ can be computed for approximately the same cost as evaluating $f$.
	\end{itemize}
	It is possible to reduce cost from $O(m^2)$ to $O(m)$!
	\begin{block}{Hutchinson's trace estimator}
		If $\bepsilon \in \bbR^m$ is a random variable with $\mathbb{E} [\bepsilon] = 0$ and $\text{Cov} (\bepsilon) = I$, then
		\vspace{-0.3cm}
		\[
		    \text{tr}(\mathbf{A}) = \text{tr}\left(\mathbf{A}\mathbb{E}_{p(\bepsilon)} \left[ \bepsilon \bepsilon^T \right]\right) =  \mathbb{E}_{p(\bepsilon)} \left[  \text{tr}\left(  \mathbf{A}  \bepsilon \bepsilon^T \right) \right] =  \mathbb{E}_{p(\bepsilon)} \left[ {\color{violet} \bepsilon^T \mathbf{A}} \bepsilon  \right]
		\]
		\vspace{-0.6 cm}
	\end{block}
	\begin{block}{FFJORD density estimation}
		\vspace{-0.8cm}
		\begin{multline*}
		    \log p(\bz(t_1)) = \log p(\bz(t_0)) - \int_{t_0}^{t_1} \text{tr}  \left( \frac{\partial f_{\btheta}(\bz(t), t)}{\partial \bz(t)} \right) dt = \\ = \log p(\bz(t_0)) - \mathbb{E}_{p(\bepsilon)} \int_{t_0}^{t_1} \left[ {\color{violet}\bepsilon^T \frac{\partial f}{\partial \bz}} \bepsilon \right] dt.
		\end{multline*}
	\end{block}
	\myfootnotewithlink{https://arxiv.org/abs/1810.01367}{Grathwohl W. et al. FFJORD: Free-form Continuous Dynamics for Scalable Reversible Generative Models, 2018} 
\end{frame}
%=======
\begin{frame}{Summary}
	\begin{itemize}
		\item Residual networks could be interpreted as solution of ODE with Euler method.
		\vfill
		\item Adjoint method generalizes backpropagation procedure and allows to train Neural ODE solving ODE for adjoint function back in time.
		\vfill
		\item Kolmogorov-Fokker-Planck theorem allows to construct continuous-in-time normalizing flow with less functional restrictions.
		\vfill
		\item FFJORD model makes such kind of NF scalable.
	\end{itemize}
\end{frame}
%=======
\end{document} 