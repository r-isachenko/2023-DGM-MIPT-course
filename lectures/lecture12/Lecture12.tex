\input{../utils/preamble}
\createdgmtitle{12}

\usepackage{tikz}

\usetikzlibrary{arrows,shapes,positioning,shadows,trees}
%--------------------------------------------------------------------------------
\begin{document}
%--------------------------------------------------------------------------------
\begin{frame}[noframenumbering,plain]
%\thispagestyle{empty}
\titlepage
\end{frame}
%=======
\begin{frame}{Recap of previous lecture}
	\vspace{-0.2cm}
	\begin{block}{SDE basics}
		Let define stochastic process $\bx(t)$ with initial condition $\bx(0) \sim p_0(\bx)$:
		\[
			d\bx = \mathbf{f}(\bx, t) dt + g(t) d \bw, 
		\]
		where $\bw(t)$ is the standard Wiener process (Brownian motion)
		\vspace{-0.2cm}
		\[		
			\bw(t) - \bw(s) \sim \cN(0, (t - s) \bI), \quad d \bw = \bepsilon \cdot \sqrt{dt}, \, \text{where } \bepsilon \sim \cN(0, \bI).
		\]
	\end{block}
	\vspace{-0.5cm}
	\begin{block}{Langevin dynamics}
		Let $\bx_0$ be a random vector. Then under mild regularity conditions for small enough $\eta$ samples from the following dynamics
		\vspace{-0.2cm}
		\[
			\bx_{t + 1} = \bx_t + \eta \frac{1}{2} \nabla_{\bx_t} \log p(\bx_t | \btheta) + \sqrt{\eta} \cdot \bepsilon, \quad \bepsilon \sim \cN(0, \bI).
		\]
		will comes from $p(\bx | \btheta)$.
	\end{block}
	The density $p(\bx | \btheta)$ is a \textbf{stationary} distribution for the Langevin SDE.
	\myfootnotewithlink{https://www.stats.ox.ac.uk/~teh/research/compstats/WelTeh2011a.pdf}{Welling M. Bayesian Learning via Stochastic Gradient Langevin Dynamics, 2011}
\end{frame}
%=======
\begin{frame}{Recap of previous lecture}
	\begin{figure}
		\centering
		\includegraphics[width=0.75\linewidth]{figs/smld}
	\end{figure}
	\vspace{-0.3cm} 
	\begin{block}{Theorem (implicit score matching)}
		\vspace{-0.6cm}
		\[
		{\color{violet}\frac{1}{2} \bbE_{\pi}\bigl\| \bs(\bx, \btheta) - \nabla_\bx \log \pi(\bx) \bigr\|^2_2} = {\color{teal}\bbE_{\pi}\Bigr[ \frac{1}{2}\| \bs(\bx, \btheta) \|_2^2 + \text{tr}\bigl(\nabla_{\bx} \bs(\bx, \btheta)\bigr) \Bigr]} + \text{const}
		\]
	\end{block}
	\vspace{-0.5cm}
	\begin{enumerate}
		\item {\color{violet}The left hand side} is intractable due to unknown $\pi(\bx)$ -- \textbf{denoising score matching}. 
		\item {\color{teal}The right hand side} is complex due to Hessian matrix -- \textbf{sliced score matching (Hutchinson's trace estimation)}.
	\end{enumerate}
	\myfootnotewithlink{https://yang-song.github.io/blog/2021/score/}{Song Y. Generative Modeling by Estimating Gradients of the Data Distribution, blog post, 2021}
\end{frame}
%=======
\begin{frame}{Recap of previous lecture}
	Let perturb original data by normal noise $p(\bx' | \bx, \sigma) = \cN(\bx' | \bx, \sigma^2 \bI)$
	\vspace{-0.3cm}
	\[
	\pi(\bx' | \sigma) = \int \pi(\bx) p(\bx' | \bx, \sigma) d\bx.
	\]
	\vspace{-0.6cm} \\
	Then the solution of 
	\vspace{-0.2cm}
	\[
	\frac{1}{2} \bbE_{\pi(\bx' | \sigma)}\bigl\| \bs(\bx', \btheta, \sigma) - \nabla_{\bx'} \log \pi(\bx' | \sigma) \bigr\|^2_2 \rightarrow \min_{\btheta}
	\]
	\vspace{-0.5cm} \\
	satisfies $\bs(\bx', \btheta, \sigma) \approx \bs(\bx', \btheta, 0) = \bs(\bx', \btheta)$ if $\sigma$ is small enough.
	\begin{block}{Theorem (denoising score matching)}
		\vspace{-0.8cm}
		\begin{multline*}
			\bbE_{\pi(\bx' | \sigma)}\bigl\| \bs(\bx', \btheta, \sigma) - \nabla_{\bx'} \log \pi(\bx' | \sigma) \bigr\|^2_2 = \\ = \bbE_{\pi(\bx)} \bbE_{p(\bx' | \bx, \sigma)}\bigl\| \bs(\bx', \btheta, \sigma) - \nabla_{\bx'} \log p(\bx' | \bx, \sigma) \bigr\|^2_2 + \text{const}(\btheta)
		\end{multline*}
		\vspace{-0.8cm}
	\end{block}
	Here $\nabla_{\bx'} \log p(\bx' | \bx, \sigma) = - \frac{\bx' - \bx}{\sigma^2}$.
	\begin{itemize}
		\item The RHS does not need to compute $\nabla_{\bx'} \log \pi(\bx' | \sigma)$ and even more $\nabla_{\bx'} \log \pi(\bx')$.
		\item $\bs(\bx', \btheta, \sigma)$ tries to \textbf{denoise} a corrupted sample.
		\item Score function $\bs(\bx', \btheta, \sigma)$ parametrized by $\sigma$. 
	\end{itemize}
	\myfootnotewithlink{http://www.iro.umontreal.ca/~vincentp/Publications/smdae_techreport.pdf}{Vincent P. A Connection Between Score Matching and Denoising Autoencoders, 2010}
\end{frame}
%=======
\begin{frame}{Recap of previous lecture}
	\begin{block}{Noise conditioned score network}
		\begin{itemize}
			\item Define the sequence of noise levels: $\sigma_1 > \sigma_2 > \dots > \sigma_L$.
			\item Train denoised score function $\bs(\bx', \btheta, \sigma)$ for each noise level:
			\vspace{-0.3cm}
			\[
				\sum_{l=1}^L {\color{violet}\sigma_l^2} \bbE_{\pi(\bx)} \bbE_{p(\bx' | \bx, \sigma_l)}\bigl\| \bs(\bx', \btheta, \sigma_l) - \nabla_\bx' \log p(\bx' | \bx, \sigma_l) \bigr\|^2_2 \rightarrow \min_{\btheta}
			\]
			\vspace{-0.5cm}
			\item Sample from \textbf{annealed} Langevin dynamics (for $l=1, \dots, L$).
		\end{itemize}
	\end{block}
	\begin{figure}
		\includegraphics[width=0.55\linewidth]{figs/multi_scale}
	\end{figure}
	\begin{figure}
		\includegraphics[width=\linewidth]{figs/duoduo}
	\end{figure}
	\myfootnotewithlink{https://arxiv.org/abs/1907.05600}{Song Y. et al. Generative Modeling by Estimating Gradients of the Data Distribution, 2019}
\end{frame}
%=======
\begin{frame}{Outline}
	\tableofcontents
\end{frame}
%=======
\section{Gaussian diffusion process}
%=======
\begin{frame}{Forward gaussian diffusion process}
	Let $\bx_0 = \bx \sim \pi(\bx)$, $\beta \in (0, 1)$. Define the Markov chain
	\[
		\bx_t = \sqrt{1 - \beta} \cdot \bx_{t - 1} + \sqrt{\beta} \cdot \bepsilon, \quad \text{where }\bepsilon \sim \cN(0, 1);
	\]
	\[
		q(\bx_t | \bx_{t-1}) = \cN(\bx_t | \sqrt{1 - \beta} \cdot \bx_{t-1}, \beta \cdot \bI).
	\]
	\vspace{-0.6cm}
	\begin{block}{Statement 1}
		Applying the Markov chain to samples from any $\pi(\bx)$ we will get $\bx_{\infty} \sim p_{\infty}(\bx) = \cN(0, 1)$. Here $p_{\infty}(\bx)$ is a \textbf{stationary} distribution:
		\vspace{-0.2cm}
		\[
			p_{\infty}(\bx) = \int q(\bx | \bx') p_{\infty}(\bx') d \bx'. 
		\]
		\vspace{-0.8cm}
	\end{block}
	\begin{block}{Statement 2}
		Denote $\bar{\alpha}_t = \prod_{s=1}^t (1 - \beta_s)$. Then 
		\vspace{-0.2cm}
		\[
			\bx_t = \sqrt{\bar{\alpha}_t} \cdot \bx_{0} + \sqrt{1 - \bar{\alpha}_t} \cdot \bepsilon, \quad \text{where } \bepsilon \sim \cN(0, 1)
		\]
		\vspace{-0.7cm}
		\[
			q(\bx_t | \bx_0) = \cN(\bx_t | \sqrt{\bar{\alpha}_t} \cdot \bx_0, (1 - \bar{\alpha}_t) \cdot \bI).
		\]
		We could sample from any timestamp using only $\bx_0$!
	\end{block}
	\myfootnotewithlink{http://proceedings.mlr.press/v37/sohl-dickstein15.pdf}{Sohl-Dickstein J. Deep Unsupervised Learning using Nonequilibrium Thermodynamics, 2015}
\end{frame}
%=======
\begin{frame}{Forward gaussian diffusion process}
	\textbf{Diffusion} refers to the flow of particles from high-density regions towards low-density regions.
	\vspace{-0.2cm}
	\begin{figure}
		\includegraphics[width=0.5\linewidth]{figs/diffusion_over_time}
	\end{figure}
	\vspace{-0.6cm}
	\begin{enumerate}
		\item $\bx_0 = \bx \sim \pi(\bx)$;
		\item $\bx_t = \sqrt{1 - \beta} \cdot \bx_{t - 1} + \sqrt{\beta} \cdot \bepsilon$, where $\bepsilon \sim \cN(0, 1)$, $t \geq 1$;
		\item $\bx_T \sim p_{\infty}(\bx) = \cN(0, 1)$, where $T >> 1$.
	\end{enumerate}
	If we are able to invert this process, we will get the way to sample $\bx \sim \pi(\bx)$ using noise samples $p_{\infty}(\bx) = \cN(0, 1)$. \\ 
	Now our goal is to revert this process.
	\myfootnotewithlink{https://ayandas.me/blog-tut/2021/12/04/diffusion-prob-models.html}{Das A. An introduction to Diffusion Probabilistic Models, blog post, 2021}
\end{frame}
%=======
\begin{frame}{Reverse gaussian diffusion process}
	\vspace{-0.3cm} 
	\begin{figure}
		\includegraphics[width=0.8\linewidth]{figs/DDPM}
	\end{figure}
	\vspace{-0.3cm} 
	Let define the reverse process
	\vspace{-0.2cm}
	\[
		p(\bx_{t - 1} | \bx_t, \btheta) = \cN \bigl(\bx_{t - 1} | \bmu(\bx_t, \btheta, t), \bsigma^2(\bx_t, \btheta, t)\bigr)
	\]
	\vspace{-0.5cm}
	\begin{minipage}{0.5\linewidth}
		\begin{block}{Forward process}
			\begin{enumerate}
				\item $\bx_0 = \bx \sim \pi(\bx)$;
				\item $\bx_t = \sqrt{1 - \beta} \cdot \bx_{t - 1} + \sqrt{\beta} \cdot \bepsilon$, where $\bepsilon \sim \cN(0, 1)$, $t \geq 1$;
				\item $\bx_T \sim p_{\infty}(\bx) = \cN(0, 1)$.
			\end{enumerate}
		\end{block}
	\end{minipage}%
	\begin{minipage}{0.5\linewidth}
		\begin{block}{Reverse process}
			\begin{enumerate}
				\item $\bx_T \sim p_{\infty}(\bx) = \cN(0, 1)$;
				\item $\bx_{t - 1} = \bsigma(\bx_t, \btheta, t) \cdot \bepsilon + \bmu(\bx_t, \btheta, t)$;
				\item $\bx_0 = \bx \sim \pi(\bx)$;
			\end{enumerate}
		\end{block}
	\end{minipage}
	\textbf{Note:} The forward process does not have any learnable parameters!
	\myfootnotewithlink{https://lilianweng.github.io/posts/2021-07-11-diffusion-models/}{Weng L. What are Diffusion Models?, blog post, 2021}
\end{frame}
%=======
\begin{frame}{Gaussian diffusion model as VAE}
	\vspace{-0.2cm}
	\begin{figure}
		\includegraphics[width=0.65\linewidth]{figs/diffusion_pgm}
	\end{figure}
	\begin{itemize}
		\item Let treat $\bz = (\bx_1, \dots, \bx_T)$ as a latent variable (\textbf{note:} each $\bx_t$ has the same size).
		\item Variational posterior distribution (\textbf{note:} there is no learnable parameters)
		\vspace{-0.4cm}
		\[
			q(\bz | \bx) = q(\bx_1, \dots, \bx_T | \bx_0) = \prod_{t = 1}^T q(\bx_t | \bx_{t - 1}).
		\]
		\vspace{-0.5cm}
		\item Probabilistic model
		\vspace{-0.2cm}
		\[
			p(\bx, \bz | \btheta) = p(\bx | \bz, \btheta) p(\bz | \btheta)
		\]
		\item Generative distribution and prior
		\vspace{-0.3cm}
		\[
			p(\bx | \bz, \btheta) = p(\bx_0 | \bx_1, \btheta); \quad 
			p(\bz | \btheta) = \prod_{t=2}^T p(\bx_{t - 1} | \bx_t, \btheta)  \cdot p(\bx_T)
		\]
	\end{itemize}
	\myfootnotewithlink{https://ayandas.me/blog-tut/2021/12/04/diffusion-prob-models.html}{Das A. An introduction to Diffusion Probabilistic Models, blog post, 2021}
\end{frame}
%=======
\section{Denoising diffusion probabilistic model (DDPM)}
%=======
\begin{frame}{Reverse gaussian diffusion process}
	\begin{figure}
		\includegraphics[width=0.8\linewidth]{figs/DDPM}
	\end{figure}
	\vspace{-0.5cm}
	\begin{block}{Forward process}
		\vspace{-0.3cm}
		\[
			q(\bx_t | \bx_{t-1}) = \cN(\bx_t | \sqrt{1 - \beta_t} \cdot \bx_{t-1}, \beta_t \cdot \bI).
		\]
		\vspace{-0.5cm}
	\end{block}
	\begin{block}{Reverse process}
		\vspace{-0.3cm}
		\[
			q(\bx_{t-1}|\bx_{t}) = \frac{q(\bx_{t}|\bx_{t-1}) {\color{violet}q(\bx_{t-1})}}{{\color{violet}q(\bx_{t})}} \approx p(\bx_{t - 1} | \bx_t, \btheta)
		\]
		\vspace{-0.3cm}
		\begin{itemize}
			\item ${\color{violet}q(\bx_{t-1})}$, ${\color{violet}q(\bx_{t})}$ are intractable.
			\item If $\beta_t$ is small enough, $q(\bx_{t-1}|\bx_{t})$ will be Gaussian (Feller, 1949).
		\end{itemize}
	\end{block}
	\myfootnotewithlink{}{Feller W. On the theory of stochastic processes, with particular reference to applications, 1949}
	\end{frame}
%=======
\begin{frame}{Reverse gaussian diffusion process}
	\begin{block}{Reverse process}
		\vspace{-0.6cm}
		\[
			q(\bx_{t-1}|\bx_{t}) \approx p(\bx_{t - 1} | \bx_t, \btheta) = \cN \bigl(\bx_{t - 1} | \bmu(\bx_t, \btheta, t), \bsigma^2(\bx_t, \btheta, t)\bigr)
		\]
		\vspace{-0.6cm}
	\end{block}
	\begin{figure}
		\includegraphics[width=0.7\linewidth]{figs/inverse_distr_1d}
	\end{figure}
	\vspace{-0.3cm}
	\begin{block}{Important distribution}
		\vspace{-0.6cm}
		\[
			q(\bx_{t-1}|\bx_{t}, \bx_0) = \frac{q(\bx_{t}|\bx_{t-1}, {\color{olive}\bx_0}) q(\bx_{t-1} | {\color{olive}\bx_0}) }{q(\bx_{t}| {\color{olive}\bx_0})} = \cN(\tilde{\bmu}_t(\bx_t, \bx_0), \tilde{\beta}_t \bI)
		\]
	\end{block}
	\myfootnotewithlink{https://arxiv.org/abs/2112.07804}{Xiao Z., Kreis K., Vahdat A. Tackling the generative learning trilemma with denoising diffusion GANs, 2021}
	\end{frame}
%=======
\subsection{Objective of DDPM}
%=======
\begin{frame}{Objective of DDPM}
	\begin{block}{ELBO}
		\vspace{-0.4cm}
		\[
			\log p(\bx | \btheta) \geq \bbE_{q({\color{teal}\bz} | \bx)} \log \frac{p(\bx, {\color{teal}\bz} | \btheta)}{q({\color{teal}\bz} | \bx)} = \cL(q, \btheta) \rightarrow \max_{q, \btheta}
		\]
		\vspace{-0.5cm}
	\end{block}
	\begin{block}{Derivation}
		{\small
		\begin{multline*}
			\cL(q, \btheta) = \bbE_{q({\color{teal}\bx_{1:T}} | \bx_0)} \log \frac{p(\bx_0, {\color{teal}\bx_{1:T}} | \btheta)}{q({\color{teal}\bx_{1:T}} | \bx_0)} =  \bbE_{q} \log \frac{{\color{violet}\prod_{t=1}^T} p(\bx_{t-1} | \bx_t, \btheta) {\color{olive}p(\bx_T)} }{{\color{violet}\prod_{t=1}^T} q(\bx_t | \bx_{t-1})}  \\ 
			= \bbE_{q} \left[ \log {\color{olive}p(\bx_T)} + \log \frac{p(\bx_0 | \bx_1, \btheta)}{q(\bx_1 | \bx_0)}  + {\color{violet}\sum_{t=2}^T} \log \left( \frac{p(\bx_{t-1} | \bx_t, \btheta)}{{\color{teal}q(\bx_t | \bx_{t-1})}}\right)\right]
		\end{multline*}
		}
		\vspace{-0.3cm}
		\[
			q(\bx_{t-1}|\bx_{t}, \bx_0) = \frac{{\color{teal} q(\bx_{t}|\bx_{t-1})} q(\bx_{t-1} |\bx_0) }{q(\bx_{t}| \bx_0)} 
		\]
		
		\[
			\log \left( \frac{p(\bx_{t-1} | \bx_t, \btheta)}{{\color{teal}q(\bx_t | \bx_{t-1})}}\right) = \log \left( \frac{p(\bx_{t-1} | \bx_t, \btheta) {\color{teal} q(\bx_{t-1}|\bx_{0})}}{{\color{teal} q(\bx_{t-1}|\bx_{t}, \bx_0) q(\bx_{t}| \bx_0)}}\right)
		\]
	\end{block}
	
	\myfootnotewithlink{https://arxiv.org/abs/2006.11239}{Ho J. Denoising Diffusion Probabilistic Models, 2020}
\end{frame}
%=======
\begin{frame}{Objective of DDPM}
	\begin{block}{Derivation}
		{\small
		\begin{multline*}
			\cL(q, \btheta) = \bbE_{q} \left[ \log p(\bx_T) + \log \frac{p(\bx_0 | \bx_1, \btheta)}{q(\bx_1 | \bx_0)}  + \sum_{t=2}^T \log \left( \frac{p(\bx_{t-1} | \bx_t, \btheta)}{{\color{teal}q(\bx_t | \bx_{t-1})}}\right)\right] \\ 
			= \bbE_{q} \biggl[  \log {\color{violet} p(\bx_T)} + \log \frac{{\color{olive}p(\bx_0 | \bx_1, \btheta)}}{q(\bx_1 | \bx_0)}  + \sum_{t=2}^T \log \left( \frac{p(\bx_{t-1} | \bx_t, \btheta)}{{\color{teal} q(\bx_{t-1}|\bx_{t}, \bx_0)}}\right) + \\ 
			+ \sum_{t=2}^T \log \left( \frac{{\color{teal} q(\bx_{t-1}|\bx_{0})}}{{\color{teal} q(\bx_{t}| \bx_0)}}\right) \biggr] = \bbE_{q} \biggl[ \log \frac{{\color{violet} p(\bx_T)}}{q(\bx_T | \bx_0)} + \log {\color{olive}p(\bx_0 | \bx_1, \btheta)} + \\
			 + \sum_{t=2}^T \log \left( \frac{p(\bx_{t-1} | \bx_t, \btheta)}{q(\bx_{t-1}|\bx_{t}, \bx_0)}\right) \biggr] = \bbE_{q} \biggl[ - KL(q(\bx_T | \bx_0) || p(\bx_T)) + \\
			 + \log p(\bx_0 | \bx_1, \btheta) - \sum_{t=2}^T  KL\bigl( q(\bx_{t-1}|\bx_{t}, \bx_0) || p(\bx_{t-1} | \bx_t, \btheta)\bigr) \biggr]
		\end{multline*}
		}
		\vspace{-0.3cm}
	\end{block}
	\myfootnotewithlink{https://arxiv.org/abs/2006.11239}{Ho J. Denoising Diffusion Probabilistic Models, 2020}
\end{frame}
%=======
\begin{frame}{Objective of DDPM}
		\vspace{-0.6cm}
		\begin{multline*}
			\cL(q, \btheta) =  \bbE_{q} \Bigl[ {\color{olive}\log p(\bx_0 | \bx_1, \btheta)} - {\color{violet}KL\bigl(q(\bx_T | \bx_0) || p(\bx_T)\bigr)} - \\
			- \sum_{t=2}^T \underbrace{KL \bigl(q(\bx_{t-1} | \bx_t, \bx_0) || p(\bx_{t - 1} | \bx_t, \btheta )\bigr)}_{\cL_t} \Bigr]
		\end{multline*}
		\vspace{-0.5cm}
	\begin{itemize}
		\item {\color{olive}First term} is a decoder distribution
		\[
			\log p(\bx_0 | \bx_1, \btheta) = \log \cN \bigl(\bx_0 | \bmu(\bx_1, \btheta, t), \bsigma^2(\bx_1, \btheta, t)\bigr)
		\] 
		\item {\color{violet}Second term} is constant ($p(\bx_T)$ is a standard Normal, $q(\bx_T | \bx_0)$ is a non-parametrical Normal).
		\item $\cL_t$ is a KL between two normal distributions:
		\[
			q(\bx_{t-1} | \bx_t, \bx_0) = \cN(\bx_{t-1} | \tilde{\bmu}_t(\bx_t, \bx_0), \tilde{\beta}_t \bI),
		\]
		$\tilde{\bmu}_t(\bx_t, \bx_0)$ and $\tilde{\beta}_t$ have analytical formulas (we omit it) and they are both dependent on $\beta_t$.
	\end{itemize}
	\myfootnotewithlink{https://arxiv.org/abs/2006.11239}{Ho J. Denoising Diffusion Probabilistic Models, 2020}
\end{frame}
%=======
\subsection{Reparametrization of DDPM}
%=======
\begin{frame}{Gaussian diffusion model as VAE}
	\vspace{-0.5cm}
	\begin{align*}
		q(\bx_{t-1} | \bx_t, \bx_0) &= \cN(\bx_{t-1} | \tilde{\bmu}_t(\bx_t, \bx_0), \tilde{\beta}_t \bI), \\
		p(\bx_{t - 1} | \bx_t, \btheta) &= \cN \bigl(\bx_{t - 1} | \bmu(\bx_t, \btheta, t), \bsigma^2(\bx_t, \btheta, t)\bigr)
	\end{align*}
	\vspace{-0.5cm}
	\begin{itemize}
		\item Assume $\bsigma^2(\bx_t, \btheta, t) = \tilde{\beta}_t \bI$.
		\item Use KL formula between two normal distributions:
		\begin{align*}
			\cL_t &= KL\Bigl(\cN\bigl(\tilde{\bmu}_t(\bx_t, \bx_0), \tilde{\beta}_t \bI \bigr) || \cN\bigl(\bmu(\bx_t, \btheta, t), \tilde{\beta}_t \bI\bigr)\Bigr) \\ 
			&= \bbE_{\bepsilon} \left[ \frac{1}{2\tilde{\beta}_t} \bigl\| \tilde{\bmu}_t(\bx_t, \bx_0) - \bmu(\bx_t, \btheta, t) \bigr\|^2 \right] \\
			& = \bbE_{\bepsilon} \left[ \frac{1}{2\tilde{\beta}_t} \left\| {\color{violet} \frac{1}{\sqrt{1 - \beta_t}}\left( \bx_t - \frac{\beta_t}{\sqrt{1 - \bar{\alpha}_t} } \bepsilon \right)} - \bmu(\bx_t, \btheta, t) \right\|^2 \right]
		\end{align*}
		 {\color{gray}Here we used the analytic expression for $\tilde{\bmu}_t(\bx_t, \bx_0)$.}
	\end{itemize}
	\begin{block}{Reparametrization}
		\vspace{-0.3cm}
		\[
			\bmu(\bx_t, \btheta, t) = \frac{1}{\sqrt{1 - \beta_t}}\left( \bx_t - \frac{\beta_t}{\sqrt{1 - \bar{\alpha}_t} } \bepsilon(\bx_t, \btheta, t) \right) 
		\]
		\vspace{-0.6cm}
	\end{block}
	\myfootnotewithlink{https://arxiv.org/abs/2006.11239}{Ho J. Denoising Diffusion Probabilistic Models, 2020}
	\end{frame}
%=======
\begin{frame}{Reparametrization of DDPM}
	\begin{block}{KL term}
		\vspace{-0.8cm}
		\begin{multline*}
			\cL_t = \bbE_{\bepsilon} \biggl[ \frac{1}{2\tilde{\beta}_t} \Bigl\| {\color{violet} \frac{1}{\sqrt{1 - \beta_t}}\left( \bx_t - \frac{\beta_t}{\sqrt{1 - \bar{\alpha}_t} } \bepsilon \right)} - \\
			- \frac{1}{\sqrt{1 - \beta_t}}\left( \bx_t - \frac{\beta_t}{\sqrt{1 - \bar{\alpha}_t} } \bepsilon(\bx_t, \btheta, t) \right)  \Bigr\|^2 \biggr] = \\
			\bbE_{\bepsilon} \left[ \frac{\beta_t^2}{2\tilde{\beta}_t (1 - \beta_t)} \left\| \frac{\bepsilon}{\sqrt{1 - \bar{\alpha}_t}} - \frac{\bepsilon({\color{teal}\bx_t}, \btheta, t)}{\sqrt{1 - \bar{\alpha}_t}}\right\|^2 \right]
		\end{multline*}
	\end{block}
		\vspace{-0.5cm}
		\[
			{\color{teal}\bx_t} = \sqrt{\bar{\alpha}_t} \cdot \bx_{0} + \sqrt{1 - \bar{\alpha}_t} \cdot \bepsilon, \quad \text{where } \bepsilon \sim \cN(0, 1)
		\]
	At each step of reverse diffusion process we try to predict the noise $\bepsilon$ that we used in forward process!
	\myfootnotewithlink{https://arxiv.org/abs/2006.11239}{Ho J. Denoising Diffusion Probabilistic Models, 2020}
\end{frame}
%=======
\begin{frame}{Summary}
	\begin{itemize}
		\item Gaussian diffusion process is a Markov chain that injects special form of Gaussian noise to the samples.
		\vfill
		\item Reverse process allows to sample from the real distribution $\pi(\bx)$ using samples from noise.
		\vfill
		\item Diffusion model is a VAE model which reverts gaussian diffusion process using variational inference.
		\vfill
		\item ELBO of DDPM is a sum of KL terms.
		\vfill
		\item At each step DDPM predicts the noise used in forward process.
	\end{itemize}
\end{frame}
\end{document} 