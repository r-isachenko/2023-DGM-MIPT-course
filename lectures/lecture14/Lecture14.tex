\input{../utils/preamble}
\createdgmtitle{14}

\usepackage{tikz}

\usetikzlibrary{arrows,shapes,positioning,shadows,trees}
%--------------------------------------------------------------------------------
\begin{document}
%--------------------------------------------------------------------------------
\begin{frame}[noframenumbering,plain]
%\thispagestyle{empty}
\titlepage
\end{frame}
%=======
\begin{frame}{Recap of previous lecture}
	\begin{figure}
		\centering
		\includegraphics[width=0.75\linewidth]{figs/smld}
	\end{figure}
	\vspace{-0.3cm} 
	\begin{block}{Theorem (implicit score matching)}
		\vspace{-0.6cm}
		\[
		{\color{violet}\frac{1}{2} \bbE_{\pi}\bigl\| \bs_{\btheta}(\bx) - \nabla_\bx \log \pi(\bx) \bigr\|^2_2} = {\color{teal}\bbE_{\pi}\Bigr[ \frac{1}{2}\| \bs_{\btheta}(\bx) \|_2^2 + \text{tr}\bigl(\nabla_{\bx} \bs_{\btheta}(\bx)\bigr) \Bigr]} + \text{const}
		\]
	\end{block}
	\vspace{-0.5cm}
	\begin{enumerate}
		\item {\color{violet}The left hand side} is intractable due to unknown $\pi(\bx)$ -- \textbf{denoising score matching}. 
		\item {\color{teal}The right hand side} is complex due to Hessian matrix -- \textbf{sliced score matching (Hutchinson's trace estimation)}.
	\end{enumerate}
	\myfootnotewithlink{https://yang-song.github.io/blog/2021/score/}{Song Y. Generative Modeling by Estimating Gradients of the Data Distribution, blog post, 2021}
\end{frame}
%=======
\begin{frame}{Recap of previous lecture}
	Let perturb original data by normal noise $p(\bx' | \bx, \sigma) = \cN(\bx' | \bx, \sigma^2 \bI)$
	\vspace{-0.3cm}
	\[
	\pi(\bx' | \sigma) = \int \pi(\bx) p(\bx' | \bx, \sigma) d\bx.
	\]
	\vspace{-0.6cm} \\
	Then the solution of 
	\vspace{-0.2cm}
	\[
	\frac{1}{2} \bbE_{\pi(\bx' | \sigma)}\bigl\| \bs_{\btheta}(\bx', \sigma) - \nabla_{\bx'} \log \pi(\bx' | \sigma) \bigr\|^2_2 \rightarrow \min_{\btheta}
	\]
	\vspace{-0.5cm} \\
	satisfies $\bs_{\btheta}(\bx', \sigma) \approx \bs(\bx', \btheta, 0) = \bs(\bx', \btheta)$ if $\sigma$ is small enough.
	\begin{block}{Theorem (denoising score matching)}
		\vspace{-0.8cm}
		\begin{multline*}
			\bbE_{\pi(\bx' | \sigma)}\bigl\| \bs_{\btheta}(\bx', \sigma) - \nabla_{\bx'} \log \pi(\bx' | \sigma) \bigr\|^2_2 = \\ = \bbE_{\pi(\bx)} \bbE_{p(\bx' | \bx, \sigma)}\bigl\| \bs_{\btheta}(\bx', \sigma) - \nabla_{\bx'} \log p(\bx' | \bx, \sigma) \bigr\|^2_2 + \text{const}(\btheta)
		\end{multline*}
		\vspace{-0.8cm}
	\end{block}
	Here $\nabla_{\bx'} \log p(\bx' | \bx, \sigma) = - \frac{\bx' - \bx}{\sigma^2}$.
	\begin{itemize}
		\item The RHS does not need to compute $\nabla_{\bx'} \log \pi(\bx' | \sigma)$ and even more $\nabla_{\bx'} \log \pi(\bx')$.
		\item $\bs_{\btheta}(\bx', \sigma)$ tries to \textbf{denoise} a corrupted sample.
		\item Score function $\bs_{\btheta}(\bx', \sigma)$ parametrized by $\sigma$. 
	\end{itemize}
	\myfootnotewithlink{http://www.iro.umontreal.ca/~vincentp/Publications/smdae_techreport.pdf}{Vincent P. A Connection Between Score Matching and Denoising Autoencoders, 2010}
\end{frame}
%=======
\begin{frame}{Recap of previous lecture}
	\begin{block}{Noise conditioned score network}
		\begin{itemize}
			\item Define the sequence of noise levels: $\sigma_1 > \sigma_2 > \dots > \sigma_L$.
			\item Train denoised score function $\bs_{\btheta}(\bx', \sigma)$ for each noise level:
			\vspace{-0.3cm}
			\[
				\sum_{l=1}^L {\color{violet}\sigma_l^2} \bbE_{\pi(\bx)} \bbE_{p(\bx' | \bx, \sigma_l)}\bigl\| \bs_{\btheta}(\bx', \sigma_l) - \nabla_\bx' \log p(\bx' | \bx, \sigma_l) \bigr\|^2_2 \rightarrow \min_{\btheta}
			\]
			\vspace{-0.5cm}
			\item Sample from \textbf{annealed} Langevin dynamics (for $l=1, \dots, L$).
		\end{itemize}
	\end{block}
	\begin{figure}
		\includegraphics[width=0.55\linewidth]{figs/multi_scale}
	\end{figure}
	\begin{figure}
		\includegraphics[width=\linewidth]{figs/duoduo}
	\end{figure}
	\myfootnotewithlink{https://arxiv.org/abs/1907.05600}{Song Y. et al. Generative Modeling by Estimating Gradients of the Data Distribution, 2019}
\end{frame}
%=======
\begin{frame}{Outline}
	\tableofcontents
\end{frame}
%=======
\section{The worst course overview}
%=======
\begin{frame}{The worst course overview :)}
	\begin{figure}
		\includegraphics[width=\linewidth]{figs/generative-overview}
	\end{figure}
	\myfootnotewithlink{https://lilianweng.github.io/posts/2021-07-11-diffusion-models/}{Weng L. What are Diffusion Models?, blog post, 2021}
\end{frame}
%=======
\begin{frame}{The worst course overview :)}
	\begin{figure}
		\includegraphics[width=0.7\linewidth]{figs/trilemma}
	\end{figure}
	\myfootnotewithlink{https://arxiv.org/abs/2112.07804}{Xiao Z., Kreis K., Vahdat A. Tackling the generative learning trilemma with denoising diffusion GANs, 2021}
\end{frame}
%=======
\begin{frame}{Summary}
	\begin{itemize}
		\item 
	\end{itemize}
\end{frame}
\end{document} 